\documentclass[a4paper,12pt]{article}
\usepackage[hmargin=1.6cm,vmargin=1.6cm]{geometry}
\usepackage{graphicx}% Include figure files
\usepackage{dcolumn}% Align table columns on decimal point
\usepackage{bm}% bold math
\usepackage{multirow} % for multirow
%\usepackage{forest}
%\usepackage{tikz-qtree,tikz-qtree-compat}
%% The amsthm package provides extended theorem environments. For example \begin{split}
\usepackage{amsthm,amsmath,amsfonts}

% set link for ebook
%\usepackage[dvipdfmx,bookmarks=true,bookmarksnumbered=false,
%            colorlinks,linkcolor=black,
%            citecolor=black,urlcolor=black]{hyperref}
\usepackage[bookmarks=true,bookmarksnumbered=false,
            colorlinks,linkcolor=black,
            citecolor=black,urlcolor=black]{hyperref}

%\usepackage{tikz}

%\nofiles
%opening
\title{Rust-based Electronic-structure Simulation Toolkit (REST)}
\author{REST development team}

\begin{document}

\maketitle



%\tableofcontents

\newpage
\section{Restricted and Unrestricted Hartree-Fock}
Refering to Szabo's book for Restricted and Unrestricted Hartree-Fock formulas 
%Close-shell expression refers to Szabo's book, Equations of 3.124-125, on Pages 140-142; 
%while open-shell expression refers to Pages 210-214
\begin{equation}
	\boldsymbol{F}\boldsymbol{C}=\boldsymbol{S}\boldsymbol{C}\varepsilon
\end{equation}
\textbf{WARNNING:}
\begin{equation}
	\sum_{j}F_{ij}^{\alpha}C_{jk}^{\alpha} = \varepsilon_k\sum_{j}S_{ij}C_{jk}^{\alpha}
\end{equation}
With the notation definition of four center integral $\left( ij|kl \right)$ of
\begin{equation}
	\left( ij|kl \right) = \int d\textbf{r}_1d\textbf{r}_2 \phi_{i}^{*}(1)\phi_{j}(1)r_{12}^{-1}\phi_{k}^{*}(2)\phi_{l}(2),
\end{equation}
the close-shell expression of Fock matrix (Similar to Equations 3.148 and 3.154 on Pages 140 and 141) is 
\begin{equation}
	\begin{split}
	F_{ij}=&H_{ij}^{core}+\sum_{kl}D_{lk}\left[\left( ij|kl \right)-\frac{1}{2}\left(il|kj  \right)\right] \\
	=&\left.H_{ij}^{core}+\sum_{kl}D_{kl}\left[\left( ij|kl \right)-\frac{1}{2}\left(ik|jl  \right)\right]\right|_{\textbf{D} \in \mathbb{R}} \\
    \end{split}
\end{equation}
The density matrix in the close shell is defined by (Equation 3.145 on Page 139)
\begin{equation}
	\begin{split}
		D_{kl}=&2\sum_{a}^{N/2}C_{ka}C_{la}^{*}\\
		=&\left.2\sum_{a}^{N/2}C_{ka}C_{la}=D_{lk}\right|_{\textbf{D}\in \mathbb{R}}\\
    %\textbf{D} = &2\textbf{C}\cdot \textbf{C}^{H}=2\textbf{C}\cdot \textbf{C}^{T}\\
    \end{split}
\end{equation}

The unrestricted expression of Fock matrix (Similar to Equations 3.348 and 3.349 on Page 214) is defined by 
\begin{equation}
	\begin{split}
	    F_{ij}^{\sigma}=&H_{ij}^{core}+\sum_{kl}\left[D_{lk}^{\textit{Tot}}\left( ij|kl \right)-D_{kl}^{\sigma}\left(il|kj  \right)\right] \\
		=&H_{ij}^{core}+\left.\sum_{kl}\left[D_{kl}^{\textit{Tot}}\left( ij|kl \right)
		    -D_{kl}^{\sigma}\left(ik|jl  \right)\right]\right|_{\textbf{D} \in \mathbb{R}} \\
	\end{split}
\end{equation}
The density matrix in the general cases (Similar to Equations 3.342-343 on Pages 213) is defined by 
\begin{equation}
	\begin{split}
		D_{kl}^{\sigma}=&\sum_{a}^{all}W_{a}^{\sigma}C_{ka}^{\sigma}C_{la}^{\sigma*}=\sum_{a}^{N_w}W_{a}^{\sigma}C_{ka}^{\sigma}C_{la}^{\sigma*}
		=\left.\sum_{a}^{N_w}W_{a}^{\sigma}C_{ka}^{\sigma}C_{la}^{\sigma}\right|_{\textbf{C}\in\mathbb{R}}\\
		\textbf{D}^{\sigma} = &\textbf{wC}^{\sigma}\cdot \textbf{C}^{\sigma H}=\textbf{wC}^{\sigma}\cdot \textbf{C}^{\sigma T}\\
		\textbf{D}^{\textit{Tot}} = &\sum_{\sigma}\textbf{D}^{\sigma}=\textbf{D}^{\alpha}+\textbf{D}^{\beta}\\
    \end{split}
\end{equation}
Here, $W_{a}^{\sigma}$ is the electron occupation number of the $a$th orbital in the $\sigma$-spin channel.
$N_w$ is the number of orbitals that have non-zero electronic occupation.

For the coulomb term, the RI expression is:
\begin{equation}
	\begin{split}
		J_{ij}^{\sigma} = &\sum_{kl}D_{kl}^{\sigma}\left( ij|kl \right)\\
		= &\sum_{kl}\sum_{\mu}D_{kl}^{\sigma}M_{ij}^{\mu}M_{kl}^{\mu}\\
		= &\sum_{\mu}M_{ij}^{\mu}\left(\sum_{kl}D_{kl}^{\sigma}M_{kl}^{\mu}  \right)\\
	\end{split}
\end{equation}

For the exchange term, the RI expression is
\begin{equation}
	\begin{split}
		K_{ij}^{\sigma} = &\sum_{kl}D_{kl}^{\sigma}\left( ik|jl \right)\\
		= &\sum_{kl}\sum_{\mu}D_{kl}^{\sigma}M_{ik}^{\mu}M_{jl}^{\mu}\\
		= &\sum_{a}^{all}W_{a}^{\sigma}\sum_{kl}\sum_{\mu}C_{ka}^{\sigma}C_{la}^{\sigma}M_{ik}^{\mu}M_{jl}^{\mu}\\
		= &\sum_{\mu}\sum_{a}^{all}W_{a}^{\sigma}\left(\sum_{k}M_{ik}^{\mu}C_{ka}\right)\left(\sum_{l}M_{jl}^{\mu}C_{la}  \right)\\
		= &\sum_{\mu}\sum_{a}^{N_w}W_{a}^{\sigma}
		\left(\sum_{k}M_{ik}^{\mu}C_{ka}^{\sigma}\right)\left(\sum_{l}M_{jl}^{\mu}C_{la}^{\sigma}  \right)\\
		= &\sum_{a}^{N_w}W_{a}^{\sigma}\sum_{\mu}B_{ia}^{\mu\sigma}B_{ja}^{\mu\sigma}\\
		= &\sum_{\mu}\sum_{a}^{N_w}W_{a}^{\sigma}B_{ia}^{\mu\sigma}B_{ja}^{\mu\sigma}
	\end{split}
\end{equation}
where $N_w$ is the number of orbitals with non-zero electronic occupations.

%The last transformation is achieved due to the Hermitian property of $M_{kl}^{\mu}$ for each $\mu$.
The total energy of Restricted Hartree-Fock is defined by (Equation 3.184 on Page 150)
\begin{equation}
	\begin{split}
		E_0 =&\left<\Psi_0\left|\hat{H}\right|\Psi_0\right>\\
		=&\frac{1}{2}\sum_{i}\sum_{j}D_{ji}\left( H_{ij}^{core}+F_{ij} \right)\\
		=&\left.\frac{1}{2}\sum_{i}\sum_{j}D_{ij}\left( H_{ij}^{core}+F_{ij} \right)\right|_{\textbf{D}\in\mathbb{R}}\\
		=&\left.\frac{1}{2}\sum_{i}\sum_{j}D_{ij}^{*}\left( H_{ij}^{core}+F_{ij} \right)\right|_{\textbf{D}\in\mathbb{C}}\\
	\end{split}
\end{equation}
However, for Unrestricted Hartree-Fock method, the total energy expression is (Exercise 3.40 on Page 215)
\begin{equation}
	\begin{split}
		E_0=&\frac{1}{2}\sum_{i}\sum_{j}\left[D_{ji}^{\textit{tot}}H_{ij}^{core}+D_{ji}^{\alpha}F_{ij}^{\alpha}+D_{ji}^{\beta}F_{ij}^{\beta} \right]\\
		   =&\left.\frac{1}{2}\sum_{i}\sum_{j}\left[D_{ij}^{\textit{tot}}H_{ij}^{core}+D_{ij}^{\alpha}F_{ij}^{\alpha}+D_{ij}^{\beta}F_{ij}^{\beta} \right]
		\right|_{\textbf{D}\in\mathbb{R}}\\
		   =&\left.\frac{1}{2}\sum_{i}\sum_{j}\left[D_{ij}^{\textit{tot}*}H_{ij}^{core}+D_{ij}^{\alpha *}F_{ij}^{\alpha}
		+D_{ij}^{\beta *}F_{ij}^{\beta} \right]\right|_{\textbf{D}\in\mathbb{C}}\\
	\end{split}
\end{equation}

\section{Restricted and Unrestricted Kohn-Sham}
Similar to the Hartree-Fock approximation, the Kohn-Sham approximation has the close-shell expression of Fock matrix
\begin{equation}
	\begin{split}
		F_{ij}^{KS}=&H_{ij}^{core}+\sum_{kl}D_{lk}\left[\left( ij|kl \right)-\frac{\alpha}{2}\left(il|kj  \right)\right]
		+\int d^3\textbf{r}\psi_i(\textbf{r})v_{xc}(\rho,\textbf{r})\psi_j(\textbf{r}) \\
	=&\left.H_{ij}^{core}+\sum_{kl}D_{kl}\left[\left( ij|kl \right)-\frac{\alpha}{2}\left(ik|jl  \right)\right]\right|_{\textbf{D} \in \mathbb{R}} 
	+V_{xc,ij} \\
    \end{split}
\end{equation}
where $\alpha$ is the mixing factor of exact exchange. For (semi-)local DFAs, like LDA, GGAs and meta-GGAs, $\alpha=0$
$V_{xc,ij}=\int d^3\textbf{r}\psi_i(\textbf{r})v_{xc}(\rho,\textbf{r})\psi_j(\textbf{r})$ has no analytic expression, which should be integrated numerically:
\begin{equation}
	\begin{split}
		V_{xc,ij} = &\int d^3\textbf{r}\psi_i(\textbf{r})\hat{v}_{xc}(\textbf{r})\psi_j(\textbf{r})\\
          = &\sum_{at}\int d^3p_{at}(\textbf{r})\psi_i(\textbf{r})\hat{v}_{xc}(\textbf{r})\psi_j(\textbf{r})\\
          = &\sum_{at,s,t}w(\textbf{r})\psi_i(\textbf{r})\hat{v}_{xc}(\textbf{r})\psi_j(\textbf{r})\\
	\end{split}
\end{equation}
where $\textbf{r}=\textbf{r}(at,s,t)$ and $w(\textbf{r})=p_{at}(\textbf{r})\cdot w_{rad}(s)\cdot w_{ang}(t)$.
The only problem remained is to determine the atom-centered radial and angular grid points and their weights.

The total energy of Restricted Kohn-Sham is defined as
\begin{equation}
	\begin{split}
		E_0 =&\left<\Psi_0\left|\hat{H}_{noxc}\right|\Psi_0\right>+E_{xc}\\
		=&\frac{1}{2}\sum_{i}\sum_{j}D_{ji}\left( H_{ij}^{core}+F_{ij}-V_{xc,ij} \right)+E_{xc}\\
		=&\left.\frac{1}{2}\sum_{i}\sum_{j}D_{ij}\left( H_{ij}^{core}+F_{ij} \right)-V_{xc}+E_{xc}\right|_{\textbf{D}\in\mathbb{R}}\\
		=&\left.\frac{1}{2}\sum_{i}\sum_{j}D_{ij}^{*}\left( H_{ij}^{core}+F_{ij}\right)-V_{xc}+E_{xc}\right|_{\textbf{D}\in\mathbb{C}}\\
	\end{split}
\end{equation}
where 
\begin{equation}
	\begin{split}
		V_{xc} = &\frac{1}{2}\sum_{i}\sum_{j}D_{ij}*V_{xc,ij}\\
	\end{split}
\end{equation}
and
\begin{equation}
	\begin{split}
		E_{xc} = &\int d^3\textbf{r}\rho(\textbf{r})\varepsilon_{xc}(\textbf{r})\\
		= &\sum_{at,s,t}w(\textbf{r})\omega(\textbf{r})\rho(\textbf{r})\varepsilon_{xc}(\textbf{r})\\
	\end{split}
\end{equation}



\section{Numerical integration and DFT}
\subsection{Basic knowledge of numerical integration for molecules}
Generally speaking, the three-dimensional molecular integrals have the form of
\begin{equation}
	I = \int d^3\textbf{r} F(\textbf{r})
\end{equation}
which can be approximated by discrete numerical summations
\begin{equation}
	I = \sum_i A_i F(\textbf{r}_i)
\end{equation}
where the $\textbf{r}_i$ and $A_i$ are discrete integration points and their respective integration weights.

In quantum chemistry, the orbitals no matter occupied or virtual features a cusp at atomic nuclei. In consequence,
a better way for this multi-center integration problem is to partition the molecular space into discrete regions
according to each atom, namely cellular approaches. In Becke's original proposal [see A. D. Becke J. Chem. Phys. 
88 (4), 15], the cellular approach was established according to a partition function $p_{at}(\textbf{r})$,
which was called \textit{relative weight function $\omega_n(\textbf{r})$} in the original paper.
This partition function was assigned to each nucleus $at$ in the system for all $\textbf{r}$ and is normalized at every $\textbf{r}$, 
\begin{equation}
	\sum_{at}p_{at}(\textbf{r}) = 1.
\end{equation}
Then any arbitrary molecular function $F(\textbf{r})$ can be decomposed into single-center components $F_n(\textbf{r})$
\begin{equation}
	F(\textbf{r})=\sum_{at}p_{at}(\textbf{r})F(\textbf{r})=\sum_{at}F_{at}(\textbf{r}).
\end{equation}
With this partition, the integration $I$ therefore reduces to a sum of single-center integrations $I_{at}$ over each of the
nuclei in the system
\begin{equation}
	\begin{split}
		I=&\int d^3\textbf{r}F(\textbf{r})=\sum_{at}\int d^3p_{at}(\textbf{r})F(\textbf{r})\\
		 =&\sum_{at}\int d^3F_{at}(\textbf{r})=\sum_{at}I_{at}\\
	 \end{split}
 \end{equation}
 The remaining problem is to design a well-behaved partition function $p_{at}(\textbf{r})$, such that each atomic sub-integration
 $I_{at}$ can be carried out using standard single-center numerical techniques in spherical polar coordinates:
 \begin{equation}
	 I_{at}=\int_0^\infty dr \int_0^{2\pi} d\theta \int_0^{\pi} d\phi F_{at}(r,\theta,\phi)r^2\sin(\theta),
 \end{equation}
 This 3D integration can be carried out by treating each of $(r,\theta,\phi)$ independently.
 The standard choice of the grids for the angular part, i.e. $(\theta\in[0,2\pi),\phi\in[0,\pi])$  with a fixed $r$ is 
 Lebedev's quadratures. In the original paper, Becke chose the Gauss-Chebyshev quadrature of the second kind for the radial
 integration (for $r\in[0,+\infty)$).

 The current version of REST makes use of the exiting library ``numgrid'' to prepare the numerical integration grids. However,
 ``numgrid'' provides only a very limiting options to generate the grids:
 \begin{itemize}
 \item Partition function $p_{at}$: only Becke's original algorithm
 \item Radial grid: 1) Lindh-Malmqvist-Gagliardi; 2) Krack-K\"oster
 \item Angular grid: Lebedev
 \end{itemize}

 \textbf{\underline{To Do List}}:
 \begin{itemize}
	 \item Partition function $p_{at}$: \\
		   1) Stratmann-Scuseria-Frisch (\url{https://doi.org/10.1016/0009-2614(96)00600-8}), which is a standard choice in many packages\\
		   2) Laqua-Kussmann-Ochsenfeld (\url{https://doi.org/10.1063/1.5049435}).
	 \item Radial grids: MultiExp (\url{https://doi.org/10.1002/jcc.10211}).
	 \item Pruned grids for SG-0, SG-1, SG-2, and SG-3 standard grids. (refer to the code in PySCF)
\end{itemize}


\subsection{Density integration}
In general, the exchange-correlation energy $E_{xc}$ of semi-local density functional approximations and 
the corresponding exchange-correlation potential $V_{xc}$ depends on density $\rho$, density gradient 
$\nabla \rho$ and kinetic density $\tau$. Therefore, we should prepare the tabulated values of these quantities 
for the numerical integration of $E_{xc}$ and $V_{xc}$.

The spin density $\rho_{\uparrow}$ in the atomic-orbital representation is shown:
\begin{equation}
	\begin{split}
		\rho_{\uparrow}(\textbf{r}) = &\sum_{a}^{occ}W_{a}^{\uparrow}|\phi_a^{\uparrow}(\textbf{r})|^2\\
		= &\sum_{a}W_{a}^{\uparrow}\sum_{ij}C_{ia}^{\uparrow}C_{ja}^{*\uparrow}\psi_i(\textbf{r})\psi_j^*(\textbf{r})\\
		= &\sum_{ij}\psi_i(\textbf{r})\psi_j^*(\textbf{r})\sum_{a}W_{a}^{\uparrow}C_{ia}^{\uparrow}C_{ja}^{*\uparrow}\\
		= &\sum_{ij}\psi_i(\textbf{r})\psi_j^*(\textbf{r})D_{ij}^{\uparrow}\\
	\end{split}
\end{equation}
In consequence, the tabulated density $P_{\uparrow,s}$ is:
\begin{equation}
	\begin{split}
		P_{\uparrow s} = &\sum_{ij}\psi_{is}D_{ij}^{\uparrow}\psi_{js}^{*}\\
		= &\sum_{ij}\psi_{is}W_{js}\\
		W = &D\cdot \psi^*\\
	\end{split}
\end{equation}
where the index $s$ is associated with the tabulated grids. The corresponding numerical integration for the total electron number is
\begin{equation}
	\begin{split}
		\int d\textbf{r}^3 \rho_{\uparrow}(\textbf{r}) = &\int d\textbf{r}^3\sum_{ij}\psi_i(\textbf{r})\psi_j^*(\textbf{r})D_{ij}^{\uparrow}\\
		= &\sum_{s}w_s\sum_{ij}\psi_i(\textbf{r}_s)\psi_j^*(\textbf{r}_s)D_{ij}^{\uparrow}\\
	\end{split}
\end{equation}
Another way to evaluate density more efficiently is
\begin{equation}
	\begin{split}
		\rho_{\uparrow s}=\rho_{\uparrow}(\textbf{r}_s) = &\sum_{a}^{occ}W_{a}^{\uparrow}|\phi_a^{\uparrow}(\textbf{r}_s)|^2\\
		= &\sum_{a}W_{a}^{\uparrow}\sum_{ij}C_{ia}^{\uparrow}C_{ja}^{*\uparrow}\psi_i(\textbf{r}_s)\psi_j^*(\textbf{r}_s)\\
		= &\sum_{a}\sum_{i}W_{a}^{\uparrow,\frac{1}{2}}C_{ia}^{\uparrow}\psi_i(\textbf{r}_s)
	    	\sum_{j}W_{a}^{\uparrow,\frac{1}{2}}C_{ja}^{*\uparrow}\psi_j^*(\textbf{r}_s)\\
	\end{split}
\end{equation}
Here, we define
\begin{equation}
	\begin{split}
	M_{as}^{\uparrow}=&\sum_{i}W_{a}^{\uparrow,\frac{1}{2}}C_{ia}^{\uparrow}\psi_{i}(\textbf{r}_s)\\
	                 =&\sum_{i}N_{ia}^{\uparrow}\psi_{is}\\
	                 =&N^{\uparrow T}\cdot \Psi\\
	\end{split}
\end{equation}
Then
\begin{equation}
	\begin{split}
		\rho_{\uparrow s}
		= &\sum_{a}\sum_{i}W_{a}^{\uparrow,\frac{1}{2}}C_{ia}^{\uparrow}\psi_i(\textbf{r}_s)
	    	\sum_{j}W_{a}^{\uparrow,\frac{1}{2}}C_{ja}^{*\uparrow}\psi_j^*(\textbf{r}_s)\\
			= &\sum_{a}M_{as}^{\uparrow} * M_{as}^{\uparrow}\\
	\end{split}
\end{equation}

The atomic-orbital basis function $\psi_i$ is a normalized GTO. At first, we consider a normalized Cartesian GTO 
$\psi_i = G(a,A,l,m,n) = N*g(a,A,l,m,n)$. Here the normalized GTO $G(a,A,l,m,n)$ and the unnormalized one $g(a,A,l,m,n)$ are defined as 
\begin{equation}
	\begin{split}
	    g(a,A,l,m,n) &= (x-x_{A})^{l}e^{-a (x-x_{A})^2}(y-y_{A})^{m}e^{-a (y-y_{A})^2}(z-z_{A})^{n}e^{-a (z-z_{A})^2}\\
	                 &= g(a,A,l)g(a,A,m)g(a,A,n)\\
	    G(a,A,l,m,n) &= G(a,A,l)G(a,A,m)G(a,A,n)\\
	\end{split}
\end{equation}
where  $g(a,A,l) = (x-x_{A})^{l} e^{-a (x-x_{A})^2}$, $G(a,A,l)=N_l g(a,A,l)$, and the normalization factor $N$ is
\begin{equation}
	N = N_{l}+N_{m}+N_{n}=\frac{1}{\sqrt{\int g^2 r^2 dr}}
    = \sqrt{\frac{2^{2l+3} (l+1)! (2a)^{l+1.5}}{(2l+2)!\sqrt{\pi}}}
\end{equation}

In consequence, a spheric GTO can be expressed as a linear combination of a set of normalized Cartesian GTOs
\begin{equation}
	G^{s}(a,A,l,m,n)=\sum_{lmn}T_{lmn}G(a,A,l,m,n)
\end{equation}
\textbf{WARNING: at present, the transition matrix $T$ is implemented for only s,p,d,f,g.}

For generalized gradient approximations (GGAs), we need 
\begin{equation}
	\begin{split}
	    \sigma[0] &= \nabla \rho_{\alpha}\cdot \nabla \rho_{\alpha}\\
	    \sigma[1] &= \nabla \rho_{\alpha}\cdot \nabla \rho_{\beta}\\
	    \sigma[2] &= \nabla \rho_{\beta} \cdot \nabla \rho_{\beta}\\
    \end{split}
\end{equation}

The key quantity is the density derivative $\nabla \rho_{\alpha}$
\begin{equation}
	\begin{split}
		\nabla \rho_{\alpha} &= \frac{\partial \rho_{\alpha}} {\partial x}
		                      + \frac{\partial \rho_{\alpha}} {\partial y}
							  + \frac{\partial \rho_{\alpha}} {\partial z} \\
							  &= \nabla_{x} \rho_{\alpha} + \nabla_{y} \rho_{\alpha} + \nabla_{z} \rho_{\alpha} 
	\end{split}
\end{equation}
where
\begin{equation}
	\begin{split}
		\nabla_x \rho_{\alpha} &= \sum_{ij}D_{ij}^{\alpha}\left[(\nabla_x \psi_i)\psi_j + \psi_i(\nabla_x \psi_j^*)\right]\\
		&= \left.\sum_{ij}D_{ij}^{\alpha}\left[(\nabla_x \psi_i)\psi_j + \psi_i(\nabla_x \psi_j)\right]\right|_{\psi\in\mathbb{R}}\\
		&= \left.2\sum_{ij}D_{ij}^{\alpha}(\nabla_x \psi_i)\psi_j\right|_{\psi\in\mathbb{R}}\\
	\end{split}
\end{equation}

Another way to evalute the density derivitative more efficiently:
\begin{equation}
	\begin{split}
		\nabla_x \rho_{\alpha} &= 2\sum_{ij}D_{ij}^{\alpha}(\nabla_x \psi_i)\psi_j\\
		&=2\sum_{ij}\sum_{a}W_{a}^{\alpha}C_{ia}^{\alpha}C_{ja}^{\alpha}(\nabla_x \psi_i)\psi_j\\
		&=2\sum_{a}W_{a}^{\alpha}\sum_{i}C_{ia}^{\alpha}(\nabla_x \psi_i)\sum_{j}C_{ja}^{\alpha}\psi_j\\
		&=2\sum_{a}\sum_{i}\sqrt{W_{a}^{\alpha}}C_{ia}^{\alpha}(\nabla_x \psi_i)\sum_{j}\sqrt{W_{a}^{\alpha}}C_{ja}^{\alpha}\psi_j\\
	\end{split}
\end{equation}

%\begin{equation}
%	\begin{split}
%		\nabla \psi_i = \frac{\partial \psi_i} {\partial x} + \frac{\partial \psi_i} {\partial y} + \frac{\partial \psi_i} {\partial z}
%	\end{split}
%\end{equation}
In consequence, 
\begin{equation}
	\begin{split}
		\sigma[0] &= \nabla \rho_{\uparrow} \cdot \nabla \rho_{\uparrow} = \sum_{i=(x,y,z)}\nabla_{i} \rho_{\uparrow}\nabla_{i}\rho_{\uparrow}\\
	\end{split}
\end{equation}

Apparently, $\{\nabla_{x}\psi_i,\nabla_{y}\psi_i,\nabla_{z}\psi_i\} $ are the new variables that should be prepared for the evaluation of $\sigma$.
\begin{equation}
	\begin{split}
	    \nabla_{x}\psi_i &= \sum_{lmn}\nabla_{x}G(a,A,l,m,n)\\
	                     &= \sum_{lmn}N[\nabla_{x}g(a,A,l)]g(a,A,m)g(a,A,n)\\
	                     &= \sum_{lmn}Ng(a,A,m)g(a,A,n)[\nabla_{x}g(a,A,l)]\\
	                     &= \sum_{lmn}Ng(a,A,m)g(a,A,n)[l*g(a,A,l-1)-2a*g(a,A,l+1)]\\
	\end{split}
\end{equation}
The last equation in the above formation is a simplification of the following derivations
\begin{equation}
	\begin{split}
		\nabla_x g(a,A,l) &= \frac{\partial}{\partial x}(x-x_A)^l e^{-a (x-x_A)^2}\\
		&=l(x-x_A)^{l-1} e^{-a (x-x_A)^2} + (x-x_A)^l e^{-a (x-x_A)^2} \frac{\partial}{\partial x}(-a (x-x_A)^2)\\
		&=l*g(a,A,l-1) -2a (x-x_A)^{l+1} e^{-a (x-x_A)^2}\\
		&=l*g(a,A,l-1) -2a*g(a,A,l+1)\\
		&=\left[l(x-x_A)^{l-1}-2a(x-x_A)^{l+1}\right] e^{-a (x-x_A)^2}\\
	\end{split}
\end{equation}

For GGAs, we put the variables \{$\rho$, $\sigma_i$ with $i=(\uparrow\uparrow, \uparrow\downarrow, \downarrow\downarrow)$\} into libxc and obtain 
\begin{equation}
	\begin{split}
	    v_\rho=\frac{\partial \varepsilon_{xc}}{\partial \rho}\\
		v_{\sigma_i}=\frac{\partial \varepsilon_{xc}}{\partial \sigma_i}\\
    \end{split}
\end{equation}

For close-shell case (spin-channel = 1)
\begin{equation}
	\begin{split}
		V_{xc,ij} = &\int d^3\textbf{r}\psi_i(\textbf{r})\hat{v}_{xc}(\textbf{r})\psi_j(\textbf{r})\\
		            &\int d^3\textbf{r}\left[
					  \psi_i(\textbf{r})\hat{v}_{\rho}(\textbf{r})\psi_j(\textbf{r}) +
					  2\psi_i(\textbf{r})(\hat{v}_{\sigma_0}(\textbf{r})\nabla\rho(\textbf{r}))\cdot\nabla\psi_j(\textbf{r}) +
					  2\psi_j(\textbf{r})(\hat{v}_{\sigma_0}(\textbf{r})\nabla\rho(\textbf{r}))\cdot\nabla\psi_i(\textbf{r})
					\right]\\
	\end{split}
\end{equation}


\subsection{Others}
\begin{equation}
	\begin{split}
		V_{xc,ij} = &\int d^3\textbf{r}\psi_i\textbf{r}\hat{v}_{xc}(\textbf{r})\psi_j(\textbf{r})
	\end{split}
\end{equation}
Following the reference of V. Blum et al. Computer Physics Communications 180 (2009) 2175-2196. (Page 9, starting from Eq. 15),
the integration can be formally divided into localized atom-centered pieces by a ``partition of unity''
\begin{equation}
	\begin{split}
		V_{xc,ij} = &\sum_{at}\int d^3p_{at}(\textbf{r})\psi_i\textbf{r}\hat{v}_{xc}(\textbf{r})\psi_j(\textbf{r})\\
	\end{split}
\end{equation}
where $p_{at}(\textbf{r})$ is the atom-centered partition function, which is defined by
\begin{equation}
	p_{at}(\textbf{r})=\frac{g_{at}(\textbf{r})}{\sum_{at'}g_{at'}(\textbf{r})}
\end{equation}
The normalizing sum over $at'$ in the denominator runs over all atoms in the system, and 
\textit{$g_{at}$ is a strongly peaked function about its originating atom}.

Under this formula, each single-atom integrand is integrated on its own grid of $N_r$ spherical integration shells 
$r(s)$ ($s=1,\dots, N_r$) with the corresponding weight of $w_{rad}(s)$.
Meanwhile the angular integration points $\Omega_t$ with tabulated integration weights $w_{ang}(t)$ are distributed so as to integrate angular
momentum functions up to a certain order exactly (often called Lebedev grids).


%About the transformation between Cartesian and spheric GTOs

Here is the transformation for l=4 and spherical harmonic GTOs $G_i(4,x)$
\begin{equation}
	\begin{split}
		G_i(4,0) = &G(0,0,4)+\frac{3}{8}(G(4,0,0)+G(0,4,0))-\frac{3\sqrt{3}}{\sqrt{35}}(G(2,0,2)+G(0,2,2)-\frac{1}{4}G(2,2,0))\\
		G_i(4,1) = &\sqrt{\frac{5}{7}}(G(1,0,3)+iG(0,1,3))
		          -\frac{3\sqrt{5}}{4\sqrt{7}}(G(3,0,1)+iG(0,3,1))-\frac{3}{4\sqrt{7}}(G(1,2,1)+iG(2,1,1))\\
		G_i(4,-1) = &\sqrt{\frac{5}{7}}(G(1,0,3)-iG(0,1,3))
		          -\frac{3\sqrt{5}}{4\sqrt{7}}(G(3,0,1)-iG(0,3,1))-\frac{3}{4\sqrt{7}}(G(1,2,1)-iG(2,1,1))\\
		G_i(4,2) = &\frac{3\sqrt{3}}{2\sqrt{14}}(G(2,0,2)-G(0,2,2))+\frac{3i}{\sqrt{14}}G(1,1,2) \\
		          &-\frac{\sqrt{5}}{4\sqrt{2}}(G(4,0,0)-G(0,4,0))-\frac{i\sqrt{5}}{2\sqrt{14}}(G(3,1,0)+G(1,3,0))\\
		G_i(4,-2) = &\frac{3\sqrt{3}}{2\sqrt{14}}(G(2,0,2)-G(0,2,2))-\frac{3i}{\sqrt{14}}G(1,1,2) \\
		          &-\frac{\sqrt{5}}{4\sqrt{2}}(G(4,0,0)-G(0,4,0))+\frac{i\sqrt{5}}{2\sqrt{14}}(G(3,1,0)+G(1,3,0))\\
		G_i(4,3) = &\frac{\sqrt{5}}{4}(G(3,0,1)-iG(0,3,1))-\frac{3}{4}(G(1,2,1)-iG(2,1,1))\\
		G_i(4,-3) = &\frac{\sqrt{5}}{4}(G(3,0,1)+iG(0,3,1))-\frac{3}{4}(G(1,2,1)+iG(2,1,1))\\
		G_i(4,4) = &\frac{\sqrt{35}}{8\sqrt{2}}(G(4,0,0)+G(0,4,0))-\frac{3\sqrt{3}}{4\sqrt{2}}G(2,2,0)+i\sqrt{\frac{5}{8}}(G(3,1,0)-G(1,3,0))\\
		G_i(4,-4) = &\frac{\sqrt{35}}{8\sqrt{2}}(G(4,0,0)+G(0,4,0))-\frac{3\sqrt{3}}{4\sqrt{2}}G(2,2,0)-i\sqrt{\frac{5}{8}}(G(3,1,0)-G(1,3,0))\\
	\end{split}
\end{equation}
The real spheric GTOs $G_r(4,x)$ are than defined as
\begin{equation}
	\begin{split}
		G_r(4,0) = & G_i(4,0)\\
		G_r(4,1) = &\frac{1}{\sqrt{2}}\left(G_i(4,1)+G_i(4,-1)\right)\\
		         = &\frac{2}{\sqrt{2}}\left(\sqrt{\frac{5}{7}}G(1,0,3)-\frac{3\sqrt{5}}{4\sqrt{7}}G(3,0,1)-\frac{3}{4\sqrt{7}}G(1,2,1)\right)\\
				 = &\sqrt{\frac{10}{7}}G(1,0,3)-\frac{3\sqrt{10}}{4\sqrt{7}}G(3,0,1)-\frac{3\sqrt{2}}{4\sqrt{7}}G(1,2,1)\\
		G_r(4,-1) = &\frac{-i}{\sqrt{2}}\left(G_i(4,1)-G_i(4,-1)\right)\\
		         = &\frac{2}{\sqrt{2}}\left(\sqrt{\frac{5}{7}}G(0,1,3)-\frac{3\sqrt{5}}{4\sqrt{7}}G(0,3,1)-\frac{3}{4\sqrt{7}}G(2,1,1)\right)\\
				 = &\sqrt{\frac{10}{7}}G(0,1,3)-\frac{3\sqrt{10}}{4\sqrt{7}}G(0,3,1)-\frac{3\sqrt{2}}{4\sqrt{7}}G(2,1,1)\\
		G_r(4,2) = &\frac{1}{\sqrt{2}}\left(G_i(4,2)+G_i(4,-2)\right)\\
		         = &\frac{2}{\sqrt{2}}\left(\frac{3\sqrt{3}}{2\sqrt{14}}(G(2,0,2)-G(0,2,2))-\frac{\sqrt{5}}{4\sqrt{2}}(G(4,0,0)-G(0,4,0))\right)\\
		         = &\frac{3\sqrt{3}}{2\sqrt{7}}(G(2,0,2)-G(0,2,2))-\frac{\sqrt{5}}{4}(G(4,0,0)-G(0,4,0))\\
		G_r(4,-2) = &\frac{-i}{\sqrt{2}}\left(G_i(4,2)-G_i(4,-2)\right)\\
		         =&\frac{-2i}{\sqrt{2}}\left(\frac{3i}{\sqrt{14}}G(1,1,2)-\frac{i\sqrt{5}}{2\sqrt{14}}(G(3,1,0)+G(1,3,0))\right)\\
		         =&\frac{3}{\sqrt{7}}G(1,1,2)-\frac{\sqrt{5}}{2\sqrt{7}}(G(3,1,0)+G(1,3,0))\\
		G_r(4,3) = &\frac{1}{\sqrt{2}}\left(G_i(4,3)+G_i(4,-3)\right)\\
		         = &\frac{2}{\sqrt{2}}\left(\frac{\sqrt{5}}{4}G(3,0,1)-\frac{3}{4}G(1,2,1)\right)\\
				 = &\frac{\sqrt{10}}{4}G(3,0,1)-\frac{3\sqrt{2}}{4}G(1,2,1)\\
		G_r(4,-3) = &\frac{-i}{\sqrt{2}}\left(G_i(4,3)-G_i(4,-3)\right)\\
		        = &\frac{-2i}{\sqrt{2}}\left(-\frac{i\sqrt{5}}{4}G(0,3,1)+\frac{3i}{4}G(2,1,1)\right)\\
				= &-\frac{\sqrt{10}}{4}G(0,3,1)+\frac{3\sqrt{2}}{4}G(2,1,1)\\
	\end{split}
\end{equation}
\begin{equation}
	\begin{split}
		G_r(4,4) = &\frac{1}{\sqrt{2}}\left(G_i(4,4)+G_i(4,-4)\right)\\
		         =&\frac{2}{\sqrt{2}}\left(\frac{\sqrt{35}}{8\sqrt{2}}(G(4,0,0)+G(0,4,0))-\frac{3\sqrt{3}}{4\sqrt{2}}G(2,2,0)\right)\\
				 =&\frac{\sqrt{35}}{8}(G(4,0,0)+G(0,4,0))-\frac{3\sqrt{3}}{4}G(2,2,0)\\
		G_r(4,-4) = &\frac{-i}{\sqrt{2}}\left(G_i(4,4)-G_i(4,-4)\right)\\
		          = &\frac{-2i}{\sqrt{2}}\left(i\sqrt{\frac{5}{8}}(G(3,1,0)-G(1,3,0))\right)\\
				  = &\frac{\sqrt{5}}{2}(G(3,1,0)-G(1,3,0))\\
	\end{split}
\end{equation}





\newpage


\section{The Fock exchange potential in reciprocal space}
The Fock exchange potential is
\begin{equation}
    \begin{split}
        V_{x}(\boldsymbol{r},\boldsymbol{r}')&=
        -e^2\sum_{\boldsymbol{q}m}2w_{\boldsymbol{q}}f_{\boldsymbol{q}m}
        \frac{\phi_{\boldsymbol{q}m}^{*}(\boldsymbol{r}')\phi_{\boldsymbol{q}m}(\boldsymbol{r})}{|\boldsymbol{r}-\boldsymbol{r}'|}\\
    \end{split}
\end{equation}
Here, $\boldsymbol{q}$ is the k point, and therefor $w_{\boldsymbol{q}}$ is the weight of the k-point $\boldsymbol{q}$. $m$ is the band index, and
therefore $f_{\boldsymbol{q}m}$ is the occupational number of the band $m$ in the k-point $\boldsymbol{q}$.

To expand the orbital $\phi_{\boldsymbol{q}m}$ in plane wave, we have
\begin{equation}
    \begin{split}
        \phi_{\boldsymbol{q}m}(\boldsymbol{r})&=\frac{1}{\sqrt{\boldsymbol{\Omega}}}\sum_{\boldsymbol{G}}
        C_{\boldsymbol{q}m}(\boldsymbol{G})e^{i(\boldsymbol{q}+\boldsymbol{G})\cdot \boldsymbol{r}}\\
    \end{split}
\end{equation}
Then the Fock exchange potential evolves 
\begin{equation}
    \begin{split}
        V_{x}(\boldsymbol{r},\boldsymbol{r}')&=-\frac{e^2}{\boldsymbol{\Omega}}\sum_{\boldsymbol{q}m}
        \frac{2w_{\boldsymbol{q}}f_{\boldsymbol{q}m}}{{|\boldsymbol{r}-\boldsymbol{r}'|}}
        \sum_{\boldsymbol{G}\boldsymbol{G}'}
        C_{\boldsymbol{q}m}^{*}(\boldsymbol{G}')e^{-i(\boldsymbol{q}+\boldsymbol{G}')\cdot \boldsymbol{r}'}
        C_{\boldsymbol{q}m}(\boldsymbol{G})e^{i(\boldsymbol{q}+\boldsymbol{G})\cdot \boldsymbol{r}}\\
    \end{split}
\end{equation}
Since, we can do the Fourier transform of the Coulomb operator as:
\begin{equation}
    \int d^3\boldsymbol{r}\frac{1}{|\boldsymbol{r}|}e^{-i\boldsymbol{q}\cdot\boldsymbol{r}}=\frac{4\pi}{|\boldsymbol{q}|^2}
\end{equation}
And the reverse Fourier transform will be:
\begin{equation}
    \begin{split}
        \frac{1}{|\boldsymbol{r}|}&=\frac{1}{(2\pi)^3}\int d^3\boldsymbol{q}\frac{4\pi}{|\boldsymbol{q}|^2}e^{i\boldsymbol{q}\cdot\boldsymbol{r}}
        =\frac{1}{2\pi^2}\int d^3\boldsymbol{q}\frac{1}{|\boldsymbol{q}|^2}e^{i\boldsymbol{q}\cdot\boldsymbol{r}}
        %=4\pi\sum\frac{1}{|\boldsymbol{q}|^2}e^{i\boldsymbol{q}\cdot\boldsymbol{r}}\\
        %&=\frac{1}{2\pi^2}\sum_{\boldsymbol{q}}\delta\boldsymbol{q}\frac{1}{|\boldsymbol{q}|^2}e^{i\boldsymbol{q}\cdot\boldsymbol{r}}\\
        %\int d^3\boldsymbol{q}\frac{4\pi}{|\boldsymbol{q}|^2}e^{i\boldsymbol{q}\cdot\boldsymbol{r}}&=\frac{1}{(2\pi)^3}\frac{1}{|\boldsymbol{r}|}\\
    \end{split}
\end{equation}
Insert this equation into the Fock exchange potential
\begin{equation}
    \begin{split}
        V_{x}(\boldsymbol{r},\boldsymbol{r}')&=-\frac{e^2}{2\pi^2\boldsymbol{\Omega}}
        \sum_{\boldsymbol{q}m}2w_{\boldsymbol{q}}f_{\boldsymbol{q}m}
        \int d^3\boldsymbol{k}
        \frac{1}{{|\boldsymbol{k}|^2}}e^{i\boldsymbol{k}\cdot(\boldsymbol{r}-\boldsymbol{r}')}\\
        &\times \sum_{\boldsymbol{G}\boldsymbol{G}'} C_{\boldsymbol{q}m}^{*}(\boldsymbol{G}')e^{-i(\boldsymbol{q}+\boldsymbol{G}')\cdot \boldsymbol{r}'}
        C_{\boldsymbol{q}m}(\boldsymbol{G})e^{i(\boldsymbol{q}+\boldsymbol{G})\cdot \boldsymbol{r}}\\
        &=-\frac{e^2}{2\pi^2\boldsymbol{\Omega}}
        \int d^3\boldsymbol{k}\sum_{\boldsymbol{G}\boldsymbol{G}'} 
        \frac{1}{{|\boldsymbol{k}|^2}}\sum_{\boldsymbol{q}m}2w_{\boldsymbol{q}}f_{\boldsymbol{q}m}\\
        &\times C_{\boldsymbol{q}m}^{*}(\boldsymbol{G}')e^{-i(\boldsymbol{k}+\boldsymbol{q}+\boldsymbol{G}')\cdot \boldsymbol{r}'}
        C_{\boldsymbol{q}m}(\boldsymbol{G})e^{i(\boldsymbol{k}+\boldsymbol{q}+\boldsymbol{G})\cdot \boldsymbol{r}}\\
    \end{split}
\end{equation}
If we make a change
\begin{equation}
    \begin{split}
        \boldsymbol{k}'&=\boldsymbol{k}+\boldsymbol{q}\\
        \boldsymbol{k}&=\boldsymbol{q}-\boldsymbol{k}'\\
    \end{split}
\end{equation}
Then
\begin{equation}
    \begin{split}
        V_{x}(\boldsymbol{r},\boldsymbol{r}')&=-\frac{e^2}{2\pi^2\boldsymbol{\Omega}}
        \int d^3\boldsymbol{k}'\sum_{\boldsymbol{G}\boldsymbol{G}'} 
        \frac{1}{{|\boldsymbol{q}-\boldsymbol{k}'|^2}}\sum_{\boldsymbol{q}m}2w_{\boldsymbol{q}}f_{\boldsymbol{q}m}\\
        &\times C_{\boldsymbol{q}m}^{*}(\boldsymbol{G}')e^{-i(\boldsymbol{k}'+\boldsymbol{G}')\cdot \boldsymbol{r}'}
        C_{\boldsymbol{q}m}(\boldsymbol{G})e^{i(\boldsymbol{k}'+\boldsymbol{G})\cdot \boldsymbol{r}}\\
        &=\int d^3\boldsymbol{k}\sum_{\boldsymbol{G}\boldsymbol{G}'}
        e^{i(\boldsymbol{k}+\boldsymbol{G})\cdot \boldsymbol{r}}e^{-i(\boldsymbol{k}+\boldsymbol{G}')\cdot \boldsymbol{r}'}\\
        &\times-\frac{e^2}{2\pi^2\boldsymbol{\Omega}}\sum_{\boldsymbol{q}m}2w_{\boldsymbol{q}}f_{\boldsymbol{q}m}
         \frac{C_{\boldsymbol{q}m}^{*}(\boldsymbol{G}')
        C_{\boldsymbol{q}m}(\boldsymbol{G})}{|\boldsymbol{k}-\boldsymbol{q}|^2}\\
        &\textcolor{red}{\textbf{?}}\\
    \end{split}
\end{equation}

%\begin{equation}
%    \begin{split}
%        RF[HSE]&=\frac{4\pi}{r^2}(1-\exp^{-r^2/(4\mu^2)})\\
%        RR[lr]&=
%    \end{split}
%\end{equation}

%\bibliographystyle{plain}
%\bibliography{note}
\section{Laplace transform of opposite-spin MP2}
The opposite-spin component of the second-order correlation energy (PT2) is written as
\begin{equation}
    \begin{split}
		E_{c}^{PT2}&=\frac{1}{N_{\boldsymbol{q}}^3}\sum_{\boldsymbol{\delta k}\boldsymbol{k}\boldsymbol{q}'}
		\sum_{ab}^{occ.}\sum_{nm}^{vir.}
		\frac{\left|\sum_{\mu}L_{an}^{\mu}(\boldsymbol{k},\boldsymbol{q})R_{bm}^{\mu}(\boldsymbol{k}',\boldsymbol{q}')\right|^2}
        {\epsilon_{a\boldsymbol{k}}+\epsilon_{b\boldsymbol{k}'}-\epsilon_{n\boldsymbol{q}}-\epsilon_{m\boldsymbol{q}'}}\\
    \end{split}
\end{equation}
For simplicity, we define $\Delta_{a\boldsymbol{k},b\boldsymbol{k}'}^{n\boldsymbol{q},m\boldsymbol{q}'}=
\epsilon_{n\boldsymbol{q}}+\epsilon_{m\boldsymbol{q}'}-\epsilon_{a\boldsymbol{k}}-\epsilon_{b\boldsymbol{k}'}$.
If we use the Laplace transformation 
\begin{equation}
	\begin{split}
	    \frac {1} {\Delta_{a\boldsymbol{k},b\boldsymbol{k}'}^{n\boldsymbol{q},m\boldsymbol{q}'}}
	    &= \int_0^{\infty}dt e^{-t\Delta_{a\boldsymbol{k},b\boldsymbol{k}'}^{n\boldsymbol{q},m\boldsymbol{q}'}}\\
		&= \sum_{q}^{N_{q}}w_{q} e^{-t_q\Delta_{a\boldsymbol{k},b\boldsymbol{k}'}^{n\boldsymbol{q},m\boldsymbol{q}'}}\\
	\end{split}
\end{equation}
to expand the opposite-spin PT2 correlation energy, we have
\begin{equation}
    \begin{split}
		E_{c}^{PT2}&=-\frac{1}{N_{\boldsymbol{q}}^3}\sum_{\boldsymbol{\delta k}\boldsymbol{k}\boldsymbol{q}'}
		\sum_{q}^{N_q}\sum_{ab}^{occ.}\sum_{nm}^{vir.}w_{q}
		\left|\sum_{\mu}L_{an}^{\mu}(\boldsymbol{k},\boldsymbol{q})R_{bm}^{\mu}(\boldsymbol{k}',\boldsymbol{q}')\right|^2
        e^{-t_q\Delta_{a\boldsymbol{k},b\boldsymbol{k}'}^{n\boldsymbol{q},m\boldsymbol{q}'}}\\
		&=-\frac{1}{N_{\boldsymbol{q}}^3}\sum_{\boldsymbol{\delta k}\boldsymbol{k}\boldsymbol{q}'}
		\sum_{q}^{N_q}\sum_{ab}^{occ.}\sum_{nm}^{vir.}
		\left|\sum_{\mu}w_{q}^{\frac 1 4}L_{an}^{\mu}(\boldsymbol{k},\boldsymbol{q})e^{-\frac{1}{2}t_q(\epsilon_{n\boldsymbol{q}}-\epsilon_{a\boldsymbol{k}})}
		w_{q}^{\frac 1 4}R_{bm}^{\mu}(\boldsymbol{k}',\boldsymbol{q}')e^{-\frac{1}{2}t_q(\epsilon_{m\boldsymbol{q}'}-\epsilon_{b\boldsymbol{k}'})}\right|^2\\
		&=-\frac{1}{N_{\boldsymbol{q}}^3}\sum_{\boldsymbol{\delta k}\boldsymbol{k}\boldsymbol{q}'}
		\sum_{q}^{N_q}\sum_{ab}^{occ.}\sum_{nm}^{vir.}
		\left|\sum_{\mu}\bar{L}_{an}^{\mu}(\boldsymbol{k},\boldsymbol{q})\bar{R}_{bm}^{\mu}(\boldsymbol{k}',\boldsymbol{q}')\right|^2\\
		&=-\frac{1}{N_{\boldsymbol{q}}^3}\sum_{\boldsymbol{\delta k}\boldsymbol{k}\boldsymbol{q}'}
		\sum_{q}^{N_q}\sum_{ab}^{occ.}\sum_{nm}^{vir.}
		\sum_{\mu v}\bar{L}_{an}^{\mu *}(\boldsymbol{k},\boldsymbol{q})\bar{L}_{an}^{v}(\boldsymbol{k},\boldsymbol{q})
		\bar{R}_{bm}^{\mu *}(\boldsymbol{k}',\boldsymbol{q}')\bar{R}_{bm}^{v}(\boldsymbol{k}',\boldsymbol{q}')\\
		&=-\frac{1}{N_{\boldsymbol{q}}^3}\sum_{\boldsymbol{\delta k}\boldsymbol{k}\boldsymbol{q}'}
		\sum_{q}^{N_q}\sum_{\mu v} 
		\sum_{an}\bar{L}_{an}^{\mu *}(\boldsymbol{k},\boldsymbol{q})\bar{L}_{an}^{v}(\boldsymbol{k},\boldsymbol{q})
		\sum_{bm}\bar{R}_{bm}^{\mu *}(\boldsymbol{k}',\boldsymbol{q}')\bar{R}_{bm}^{v}(\boldsymbol{k}',\boldsymbol{q}')\\
		&=-\frac{1}{N_{\boldsymbol{q}}^3}\sum_{\boldsymbol{\delta k}\boldsymbol{k}\boldsymbol{q}'}
		\sum_{q}^{N_q}\sum_{\mu v} 
		\bar{M}_{\mu v}(\boldsymbol{k},\boldsymbol{q})\bar{N}_{\mu v}(\boldsymbol{k}',\boldsymbol{q}')
    \end{split}
\end{equation}
with 
\begin{equation}
	\begin{split}
		\bar{L}_{an}^{\mu}(\boldsymbol{k},\boldsymbol{q})
		&=w_{q}^{\frac 1 4}L_{an}^{\mu}(\boldsymbol{k},\boldsymbol{q})e^{-\frac{1}{2}t_q(\epsilon_{n\boldsymbol{q}}-\epsilon_{a\boldsymbol{k}})}\\
		\bar{R}_{bm}^{\mu}(\boldsymbol{k}',\boldsymbol{q}')
		&=w_{q}^{\frac 1 4}R_{bm}^{\mu}(\boldsymbol{k}',\boldsymbol{q}')e^{-\frac{1}{2}t_q(\epsilon_{m\boldsymbol{q}'}-\epsilon_{b\boldsymbol{k}'})}\\
		\bar{M}_{\mu v}(\boldsymbol{k},\boldsymbol{q})
		&=\sum_{an}\bar{L}_{an}^{\mu *}(\boldsymbol{k},\boldsymbol{q})\bar{L}_{an}^{v}(\boldsymbol{k},\boldsymbol{q})\\
		&=\sum_{an}w^{\frac 1 2}e^{-t_q(\epsilon_{n\boldsymbol{q}}-\epsilon_{a\boldsymbol{k}})}
		L_{an}^{\mu *}(\boldsymbol{k},\boldsymbol{q})L_{an}^{v}(\boldsymbol{k},\boldsymbol{q})\\
		\bar{N}_{\mu v}(\boldsymbol{k}',\boldsymbol{q}')
		&=\sum_{bm}\bar{R}_{bm}^{\mu *}(\boldsymbol{k}',\boldsymbol{q}')\bar{R}_{bm}^{v}(\boldsymbol{k}',\boldsymbol{q}')\\
		&=\sum_{bm}w^{\frac 1 2}e^{-t_q(\epsilon_{m\boldsymbol{q}'}-\epsilon_{b\boldsymbol{k}'})}
		R_{bm}^{\mu *}(\boldsymbol{k}',\boldsymbol{q}')R_{bm}^{v}(\boldsymbol{k}',\boldsymbol{q}')\\
	\end{split}
\end{equation}


\section{Memory distribution for periodic-PT2}

\end{document}

